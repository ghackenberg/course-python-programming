\documentclass[aspectratio=169,8pt]{beamer}

% Loading the custom Beamer theme
\newcommand{\basepath}{../../../latex-beamer-theme-fhooe/sources}
\usepackage{\basepath/beamerthemefhooe}

\renewcommand{\sectiontocframesubtitle}{Table of contents}

\title{Programming with Python}
\subtitle{Session 9: Error Handling and Debugging}
\author[Dr. Georg Hackenberg, Professor for Industrial Informatics (\href{mailto:georg.hackenberg@fh-wels.at}{georg.hackenberg@fh-wels.at})]{Dr. Georg Hackenberg BSc MSc\\Professor for Industrial Informatics\\(\href{mailto:georg.hackenberg@fh-wels.at}{georg.hackenberg@fh-wels.at})}
\institute[Subject Area Information Technology, School of Engineering, University of Applied Sciences Upper Austria]{Subject Area Information Technology\\School of Engineering\\University of Applied Sciences Upper Austria}
\date{\today}

\begin{document}

\begin{frame}
    \titlepage
\end{frame}

\begin{frame}{Learning Objectives}
    \begin{itemize}
        \item Handle exceptions with `try-except` blocks.
        \item Understand different types of errors.
        \item Use a debugger to find and fix errors.
        \item Write robust and error-resistant code.
    \end{itemize}
\end{frame}

\section{Exceptions}
\subsection{try-except}
\subsection{Handling Multiple Exceptions}
\subsection{finally}
\subsection{Raising Exceptions}

\section{Debugging}
\subsection{Using a Debugger}
\subsection{Setting Breakpoints}
\subsection{Inspecting Variables}

\end{document}
